\documentclass[12pt,a4paper]{article}
\usepackage{authblk}
\usepackage[margin=1in]{geometry}
\usepackage{fancyhdr}

\pagestyle{fancy}
\begin{document}

\title{GPT Pedagogy}
\author[1]{Matthew Pisano}
\author[1]{Tiburon Benavides}
\author[1]{Lu Zhou}
\author[1]{Yanshen Lin}
\author[1]{Yiyang Cai}
\affil[1]{Rensselaer Polytechnic Institute}
\date{\today}

\maketitle

\abstract{
    GPT-3 is a Large Language Model which, through overuse, is often used to deprive students of the 
    opportunity to learn effectively. Our intended use case of GPT enables the 
    transparent use of the technology between instructor and student. We aim to create a 
    more active and participatory learning environment through the usage of the model in active
    learning. Our long-term goal for higher education is along the lines of the fourth UN SDG:
    
    \begin{quote}
        Ensure inclusive and equitable quality education and promote lifelong 
        learning opportunities for all.
    \end{quote}

    We view GPT as a way to create an adaptive learning experience which 
    promotes the educational endeavor rather than detracts from it.

    Our plan is to develop a GPT-3 based learning assistant for the \textit{Introduction to Biology}
    course here at RPI.  Through a partnership with the professor, we will gather sufficient amounts
    of course material to fine-tune a pre-trained GPT-3 model. This will create a knowledgeable 
    and focused learning assistant.  By creating this personalized AI tutor we can make it easier 
    and more transparent for students to learn the material and reduce their stress.

    One of our goals is for the model to maintain its conversational abilities while embedding 
    additional knowledge about faculty defined key learning objectives. The model will generate a 
    series of topic-relevant questions, evaluate the answers of those questions, and give 
    useful feedback or counter-examples to the student. The model benefits from human-in-the-loop 
    reinforcement learning by storing previous chats from students and faculty alike.

    We also aim for this model to be generalizable to other classes and disciplines in the future.
    The model will work especially well for courses where it is difficult to give personalized 
    feedback to each learner in each class meeting time, such as in classes that have a high
    student to faculty ratio. It will also work well for courses with students interested in AI 
    who cannot adequately engage with that interest through the course material.
}


\end{document}
