\documentclass[12pt,a4paper]{article}
\usepackage{authblk}
\usepackage[margin=1in]{geometry}
\usepackage{fancyhdr}

\pagestyle{fancy}

\title{GPT Pedagogy}
\author[1]{Matthew Pisano}
\author[1]{Tiburon Benavides}
\author[1]{Lu Zhou}
\author[1]{Yanshen Lin}
\author[1]{Yiyang Cai}
\affil[1]{Rensselaer Polytechnic Institute}
\date{\today}

\begin{document}
    \maketitle

    \abstract{
        GPT-3 is a Large Language Model which, through overuse, is often used to deprive students of the
        opportunity to learn effectively. Our intended use case of GPT enables the
        transparent use of the technology between instructor and student. We aim to create a
        more active and participatory learning environment through the usage of the model in active
        learning. Our long-term goal for higher education is along the lines of the fourth UN SDG:

        \begin{quote}
            Ensure inclusive and equitable quality education and promote lifelong
            learning opportunities for all.
        \end{quote}

        We view GPT as a way to create an adaptive learning experience which promotes the educational
        endeavor rather than detracts from it.

        Our plan is to develop a GPT-3 based learning assistant for the \textit{Introduction to Biology}
        course here at RPI.  Through a partnership with the professor, we will gather sufficient amounts
        of course material to fine-tune a pre-trained GPT-3 model. This will create a knowledgeable
        and focused learning assistant, \textit{Mathesis} (from the ancient greek work for learning).
        By creating this personalized AI tutor we can make it easier and more transparent for
        students to learn the material and reduce their stress.

        One of our goals is for the model to maintain its conversational abilities while embedding
        additional knowledge about faculty defined key learning objectives. \textit{Mathesis} will
        generate a series of topic-relevant questions, evaluate the answers of those questions, and give
        useful feedback or counter-examples to the student. The model benefits from human-in-the-loop
        reinforcement learning by storing previous chats from students and faculty alike.

        We also aim for this model to be generalizable to other classes and disciplines in the future.
        The model will work especially well for courses where it is difficult to give personalized
        feedback to each learner in each class meeting time, such as in classes that have a high
        student to faculty ratio. It will also work well for courses with students interested in AI
        who cannot adequately engage with that interest through the course material.
    }

    \pagebreak
    \section{Introduction}

    The idea for this project was conceived before the beginning of this project in the context of
    our Informatics class.  It stemmed from observations about the current trend that large language
    models are heading in and how many educational institutions are not fully prepared to handle the
    fallout of this trend.  Our primary motivation for this project is to remedy this issue.  By
    providing students with an opportunity to use large language models, like GPT, as a learning tool,
    we allow students to gain valuable insight into the practical benefits and shortcomings of these models.

    \section{Project Use Case}

    \subsection{Specifications}

    For the development of the use case for this project, it is important to detail several
    specifications that describe GPT Pedagogy in a more rigorous manner.  One of these specifications
    is the list of the functional requirements of the project.  To fulfil our objectives, the
    project needs to:

    \begin{itemize}
        \item Provide students with a knowledgeable teacher that can answer any course-relevant question
        \item Be able to automatically generate a series of evaluation questions for students and
        reference answers based on the core topics of the course
        \item Receive students' answers to evaluation questions and evaluate their correctness based
        off of the reference answers
        \item Provide useful feedback to students if their answers did not sufficiently match the
        reference answers
        \item Allow instructors to review and regenerate all questions generated by the model
        \item Allow instructors to evaluate a summary of student performance based off of the
        automatically generated questions
    \end{itemize}

    Another important specification to consider is that of the entropy/uncertainty of the project.
    Overall, we have worked to minimize the total uncertainty of the project.  It is important to note,
    however, that the GPT model will always introduce some amount of extra uncertainty.

    One of the areas where we worked to minimize the uncertainty in was the user interface.  This is
    split into two parts, the student and administrator views.  In both of these cases, our design
    was oriented towards a simple, straightforward interface.  We worked to accomplish this through
    the use of a minimalist design.  Users can choose to choose only a few tabs: the main chat and
    the lesson evaluations.  The administrators have a similar view, with th addition of the ability
    to edit the questions that are displayed to students.  In our design, buttons, animations, and
    images are kept to a minimum.  This will hopefully allow users to focus more on working on
    evaluations or interacting with \textit{Mathesis} to better learn specific topics.

    Another way we worked to minimize uncertainty is through the fine-tuning of the model on our
    own, customized, training data.  By feeding \textit{Mathesis} significant amounts of course
    material, we have narrowed its focus down to the relevant topics of the course.  While the model
    does not forget its previously learned knowledge, its ability to correctly interpret and recall
    information related to the course has increased.  This lowers the uncertainty of the project
    by encouraging the trained model to generate responses that are targeted towards the course
    material.  This will increase its helpfulness to students as unrelated responses may provide
    misinformation or serve to demotivate students from interacting with the system.

    The use case that we developed addresses what we expect to be the basic low of the system, along
    with any reasonable alternate or exception flows.  The goal of this system is the same as the
    project in general: to use an interface and pre-trained model to provide students with a flexible
    and helpful learning assistant for the \textit{Introduction to Biology} course at RPI.

    We have included the particular flows that we did in order to properly scope our use case.  Our intention
    is to cover all relevant ways in which students and administrators, while not including flows that
    may bloat our design of this early use case.  Examples of such flows would be deliberate
    adversarial attacks to the system.  While these attacks are a near certainty in a deployed system,
    the time constraints on this project has not allowed us to account for them in our working prototype.
    Due to this reason, we have set this and similar edge cases to be outside our scope.

    \subsection{Implementation Considerations}
    ...

    \section{Architectural Models}

    \subsection{Conceptual Model}
    ...
    \subsection{Logical Model}
    ...

    \subsection{Implementation}
    ...

    \section{Conclusion}

    \subsection{Future Work}
    In our future work in this project, we expect to both finalize the attributes of the project that
    we have included in its current scope and to expand the projects scope to cover more edge cases.

    ...

    Ways in which we would expand the scope of the project would to be to better handle reasonable
    edge cases.  These could come in the form of students needing help with navigating the new system,
    handling adversarial attacks from students, or further automating the process of training models
    on new datasets.  These possible additions would serve to help students better adapt to using the
    \textit{Mathesis} assistant and its surrounding learning platform, deter students from using
    the model for unintended purposes, and creating encouraging environment for other professors who
    would like to include their courses into the learning system in future semesters respectively.

    \subsection{Closing Thoughts}
    ...

    \pagebreak
    % references and bib
\end{document}
